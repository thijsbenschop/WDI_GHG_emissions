% Options for packages loaded elsewhere
\PassOptionsToPackage{unicode}{hyperref}
\PassOptionsToPackage{hyphens}{url}
\PassOptionsToPackage{dvipsnames,svgnames,x11names}{xcolor}
%
\documentclass[
  letterpaper,
  DIV=11,
  numbers=noendperiod]{scrartcl}

\usepackage{amsmath,amssymb}
\usepackage{iftex}
\ifPDFTeX
  \usepackage[T1]{fontenc}
  \usepackage[utf8]{inputenc}
  \usepackage{textcomp} % provide euro and other symbols
\else % if luatex or xetex
  \usepackage{unicode-math}
  \defaultfontfeatures{Scale=MatchLowercase}
  \defaultfontfeatures[\rmfamily]{Ligatures=TeX,Scale=1}
\fi
\usepackage{lmodern}
\ifPDFTeX\else  
    % xetex/luatex font selection
\fi
% Use upquote if available, for straight quotes in verbatim environments
\IfFileExists{upquote.sty}{\usepackage{upquote}}{}
\IfFileExists{microtype.sty}{% use microtype if available
  \usepackage[]{microtype}
  \UseMicrotypeSet[protrusion]{basicmath} % disable protrusion for tt fonts
}{}
\makeatletter
\@ifundefined{KOMAClassName}{% if non-KOMA class
  \IfFileExists{parskip.sty}{%
    \usepackage{parskip}
  }{% else
    \setlength{\parindent}{0pt}
    \setlength{\parskip}{6pt plus 2pt minus 1pt}}
}{% if KOMA class
  \KOMAoptions{parskip=half}}
\makeatother
\usepackage{xcolor}
\setlength{\emergencystretch}{3em} % prevent overfull lines
\setcounter{secnumdepth}{-\maxdimen} % remove section numbering
% Make \paragraph and \subparagraph free-standing
\ifx\paragraph\undefined\else
  \let\oldparagraph\paragraph
  \renewcommand{\paragraph}[1]{\oldparagraph{#1}\mbox{}}
\fi
\ifx\subparagraph\undefined\else
  \let\oldsubparagraph\subparagraph
  \renewcommand{\subparagraph}[1]{\oldsubparagraph{#1}\mbox{}}
\fi


\providecommand{\tightlist}{%
  \setlength{\itemsep}{0pt}\setlength{\parskip}{0pt}}\usepackage{longtable,booktabs,array}
\usepackage{calc} % for calculating minipage widths
% Correct order of tables after \paragraph or \subparagraph
\usepackage{etoolbox}
\makeatletter
\patchcmd\longtable{\par}{\if@noskipsec\mbox{}\fi\par}{}{}
\makeatother
% Allow footnotes in longtable head/foot
\IfFileExists{footnotehyper.sty}{\usepackage{footnotehyper}}{\usepackage{footnote}}
\makesavenoteenv{longtable}
\usepackage{graphicx}
\makeatletter
\def\maxwidth{\ifdim\Gin@nat@width>\linewidth\linewidth\else\Gin@nat@width\fi}
\def\maxheight{\ifdim\Gin@nat@height>\textheight\textheight\else\Gin@nat@height\fi}
\makeatother
% Scale images if necessary, so that they will not overflow the page
% margins by default, and it is still possible to overwrite the defaults
% using explicit options in \includegraphics[width, height, ...]{}
\setkeys{Gin}{width=\maxwidth,height=\maxheight,keepaspectratio}
% Set default figure placement to htbp
\makeatletter
\def\fps@figure{htbp}
\makeatother

\usepackage{booktabs}
\usepackage{longtable}
\usepackage{array}
\usepackage{multirow}
\usepackage{wrapfig}
\usepackage{float}
\usepackage{colortbl}
\usepackage{pdflscape}
\usepackage{tabu}
\usepackage{threeparttable}
\usepackage{threeparttablex}
\usepackage[normalem]{ulem}
\usepackage{makecell}
\usepackage{xcolor}
\KOMAoption{captions}{tableheading}
\makeatletter
\makeatother
\makeatletter
\makeatother
\makeatletter
\@ifpackageloaded{caption}{}{\usepackage{caption}}
\AtBeginDocument{%
\ifdefined\contentsname
  \renewcommand*\contentsname{Table of contents}
\else
  \newcommand\contentsname{Table of contents}
\fi
\ifdefined\listfigurename
  \renewcommand*\listfigurename{List of Figures}
\else
  \newcommand\listfigurename{List of Figures}
\fi
\ifdefined\listtablename
  \renewcommand*\listtablename{List of Tables}
\else
  \newcommand\listtablename{List of Tables}
\fi
\ifdefined\figurename
  \renewcommand*\figurename{Figure}
\else
  \newcommand\figurename{Figure}
\fi
\ifdefined\tablename
  \renewcommand*\tablename{Table}
\else
  \newcommand\tablename{Table}
\fi
}
\@ifpackageloaded{float}{}{\usepackage{float}}
\floatstyle{ruled}
\@ifundefined{c@chapter}{\newfloat{codelisting}{h}{lop}}{\newfloat{codelisting}{h}{lop}[chapter]}
\floatname{codelisting}{Listing}
\newcommand*\listoflistings{\listof{codelisting}{List of Listings}}
\makeatother
\makeatletter
\@ifpackageloaded{caption}{}{\usepackage{caption}}
\@ifpackageloaded{subcaption}{}{\usepackage{subcaption}}
\makeatother
\makeatletter
\@ifpackageloaded{tcolorbox}{}{\usepackage[skins,breakable]{tcolorbox}}
\makeatother
\makeatletter
\@ifundefined{shadecolor}{\definecolor{shadecolor}{rgb}{.97, .97, .97}}
\makeatother
\makeatletter
\makeatother
\makeatletter
\makeatother
\ifLuaTeX
  \usepackage{selnolig}  % disable illegal ligatures
\fi
\IfFileExists{bookmark.sty}{\usepackage{bookmark}}{\usepackage{hyperref}}
\IfFileExists{xurl.sty}{\usepackage{xurl}}{} % add URL line breaks if available
\urlstyle{same} % disable monospaced font for URLs
\hypersetup{
  pdftitle={WDI Greenhouse Gas Emissions indicators},
  pdfauthor={Thijs Benschop},
  colorlinks=true,
  linkcolor={blue},
  filecolor={Maroon},
  citecolor={Blue},
  urlcolor={Blue},
  pdfcreator={LaTeX via pandoc}}

\title{WDI Greenhouse Gas Emissions indicators}
\author{Thijs Benschop}
\date{11/17/23}

\begin{document}
\maketitle
\ifdefined\Shaded\renewenvironment{Shaded}{\begin{tcolorbox}[borderline west={3pt}{0pt}{shadecolor}, sharp corners, enhanced, breakable, boxrule=0pt, interior hidden, frame hidden]}{\end{tcolorbox}}\fi

This note describes the current greenhouse gas (GHG) emissions
indicators in the WDI and proposes a new set of indicators using a
different data source. The indicators are evaluated using the inclusion
criteria comparing the current and proposed indicators.

\hypertarget{current-ghg-emissions-indicators-in-wdi}{%
\subsection{Current GHG emissions indicators in
WDI}\label{current-ghg-emissions-indicators-in-wdi}}

Currently, 37 indicators on GHG emissions are included in the WDI. The
proposal to revise the set of indicators and the data source is due to
several shortcomings of the current set of indicators:

\begin{itemize}
\tightlist
\item
  many indicators are outdated and no longer updated by source (or not
  available - IEA data)
\item
  many indicators are not reported annually and/or do not start in 1960
\item
  indicators from different sources cannot be compared due to different
  definitions/units and methodologies
\item
  country coverage for some indicators is low
\end{itemize}

The following 37 GHG related indicators are currently in the WDI
database:

\begin{tabu} to \linewidth {>{\raggedright}X>{\raggedright}X>{\raggedright}X>{\raggedright}X>{\raggedright}X>{\raggedright}X}
\hline
Series code & Description & GHG & Sector & Source & Time coverage\\
\hline
EN.ATM.CO2E.KD.GD & CO2 emissions (kg per 2017 US\$ of GDP) & CO2 & All & CAIT & 1990-2020\\
\hline
EN.ATM.CO2E.KT & Total CO2 emissions (thousand metric tons of CO2 excluding Land-Use Change and Forestry) & CO2 & All & CAIT & 1990-2020\\
\hline
EN.ATM.CO2E.PC & CO2 emissions (metric tons per capita) & CO2 & All & CAIT & 1990-2020\\
\hline
EN.ATM.CO2E.PP.GD & CO2 emissions (kg per PPP \$ of GDP) & CO2 & All & CAIT & 1990-2020\\
\hline
EN.ATM.CO2E.PP.GD.KD & CO2 emissions (kg per 2017 PPP \$ of GDP) & CO2 & All & CAIT & 1990-2020\\
\hline
EN.ATM.GHGT.KT.CE & Total greenhouse gas emissions (thousand metric tons of CO2 equivalent excluding Land-Use Change and Forestry) & All & All & CAIT & 1990-2020\\
\hline
EN.ATM.METH.KT.CE & Total methane emissions (thousand metric tons of CO2 equivalent excluding Land-Use Change and Forestry) & CH4 & All & CAIT & 1990-2020\\
\hline
EN.ATM.NOXE.KT.CE & Total nitrous oxide emissions (thousand metric tons of CO2 equivalent excluding Land-Use Change and Forestry) & N2O & All & CAIT & 1990-2020\\
\hline
EN.ATM.METH.AG.KT.CE & Agricultural methane emissions (thousand metric tons of CO2 equivalent) & CH4 & Agriculture & CAIT & 1990-2020\\
\hline
EN.ATM.NOXE.AG.KT.CE & Agricultural nitrous oxide emissions (thousand metric tons of CO2 equivalent) & N2O & Agriculture & CAIT & 1990-2020\\
\hline
EN.ATM.METH.EG.KT.CE & Methane emissions in energy sector (thousand metric tons of CO2 equivalent) & CH4 & Energy & CAIT & 1990-2020\\
\hline
EN.ATM.NOXE.EG.KT.CE & Nitrous oxide emissions in energy sector (thousand metric tons of CO2 equivalent) & N2O & Energy & CAIT & 1990-2020\\
\hline
EN.ATM.CO2E.EG.ZS & CO2 intensity (kg per kg of oil equivalent energy use) & CO2 & Energy & CDIAC & 1990-2015\\
\hline
EN.ATM.CO2E.GF.KT & CO2 emissions from gaseous fuel consumption (kt) & CO2 & Energy & CDIAC & 1960-2016\\
\hline
EN.ATM.CO2E.GF.ZS & CO2 emissions from gaseous fuel consumption (\% of total) & CO2 & Energy & CDIAC & 1960-2016\\
\hline
EN.ATM.CO2E.LF.KT & CO2 emissions from liquid fuel consumption (kt) & CO2 & Energy & CDIAC & 1960-2016\\
\hline
EN.ATM.CO2E.LF.ZS & CO2 emissions from liquid fuel consumption (\% of total) & CO2 & Energy & CDIAC & 1960-2016\\
\hline
EN.ATM.CO2E.SF.KT & CO2 emissions from solid fuel consumption (kt) & CO2 & Energy & CDIAC & 1960-2016\\
\hline
EN.ATM.CO2E.SF.ZS & CO2 emissions from solid fuel consumption (\% of total) & CO2 & Energy & CDIAC & 1960-2016\\
\hline
EN.ATM.GHGO.KT.CE & Other greenhouse gas emissions, HFC, PFC and SF6 (thousand metric tons of CO2 equivalent) & Fluorinated gases & All & EDGAR (EC JRC/PBL) & 1970-2016 (not annual)\\
\hline
EN.ATM.GHGO.ZG & Other greenhouse gas emissions (\% change from 1990) & Fluorinated gases & All & EDGAR (EC JRC/PBL) & 1991-2012\\
\hline
EN.ATM.GHGT.ZG & Total greenhouse gas emissions (\% change from 1990) & All & All & EDGAR (EC JRC/PBL) & 1991-2012\\
\hline
EN.ATM.HFCG.KT.CE & HFC gas emissions (thousand metric tons of CO2 equivalent) & Fluorinated gases & All & EDGAR (EC JRC/PBL) & 1990-2010 (not annual)\\
\hline
EN.ATM.METH.ZG & Methane emissions (\% change from 1990) & CH4 & All & EDGAR (EC JRC/PBL) & 1991-2012\\
\hline
EN.ATM.NOXE.ZG & Nitrous oxide emissions (\% change from 1990) & N2O & All & EDGAR (EC JRC/PBL) & 1991-2012\\
\hline
EN.ATM.PFCG.KT.CE & PFC gas emissions (thousand metric tons of CO2 equivalent) & Fluorinated gases & All & EDGAR (EC JRC/PBL) & 1990-2010 (not annual)\\
\hline
EN.ATM.SF6G.KT.CE & SF6 gas emissions (thousand metric tons of CO2 equivalent) & Fluorinated gases & All & EDGAR (EC JRC/PBL) & 1990-2010 (not annual)\\
\hline
EN.ATM.NOXE.AG.ZS & Agricultural nitrous oxide emissions (\% of total) & N2O & Agriculture & EDGAR (EC JRC/PBL) & 1970-2008\\
\hline
EN.ATM.METH.AG.ZS & Agricultural methane emissions (\% of total) & CH4 & Energy & EDGAR (EC JRC/PBL) & 1970-2008\\
\hline
EN.ATM.METH.EG.ZS & Energy related methane emissions (\% of total) & CH4 & Energy & EDGAR (EC JRC/PBL) & 1970-2008\\
\hline
EN.ATM.NOXE.EG.ZS & Nitrous oxide emissions in energy sector (\% of total) & N2O & Energy & EDGAR (EC JRC/PBL) & 1970-2008\\
\hline
EN.CO2.BLDG.ZS & CO2 emissions from residential buildings and commercial and public services (\% of total fuel combustion) & CO2 & Other & IEA & 1960-2014\\
\hline
EN.CO2.ETOT.ZS & CO2 emissions from electricity and heat production, total (\% of total fuel combustion) & CO2 & Other & IEA & 1960-2014\\
\hline
EN.CO2.MANF.ZS & CO2 emissions from manufacturing industries and construction (\% of total fuel combustion) & CO2 & Other & IEA & 1960-2014\\
\hline
EN.CO2.OTHX.ZS & CO2 emissions from other sectors, excluding residential buildings and commercial and public services (\% of total fuel combustion) & CO2 & Other & IEA & 1960-2014\\
\hline
EN.CO2.TRAN.ZS & CO2 emissions from transport (\% of total fuel combustion) & CO2 & Other & IEA & 1960-2014\\
\hline
EN.CLC.GHGR.MT.CE & GHG net emissions removals by LUCF (Mt of CO2 equivalent) & All & LUCF & UNFCCCC & 1990-2009\\
\hline
\end{tabu}

\hypertarget{list-of-proposed-indicators}{%
\subsection{List of proposed
indicators}\label{list-of-proposed-indicators}}

Based on the current selection of indicators included in the WDI as well
as the available time series in the PRIMAP-hist dataset and the
relevance of indicators to the WDI audience, it is proposed to include
the following set of indicators in the WDI. All indicators have data
from 1960 until 2022. For some countries and gases the values for 2022
are extrapolated.

\begin{tabu} to \linewidth {>{\raggedleft}X>{\raggedright}X>{\raggedright}X>{\raggedright}X>{\raggedright}X>{\raggedright}X}
\hline
Number & Indicator Code & Indicator Name & Gases & Sector & Type\\
\hline
1 & EN.GHG.TOTL.KT.CE & Greenhouse gas emissions: All Kyoto Gases (Total excluding LULUCF) & KYOTOGHG (AR4GWP100) & National Total excluding LULUCF & absolute emissions\\
\hline
2 & EN.GHG.CO2E.KT & Greenhouse gas emissions: Carbon Dioxide (CO2) (Total excluding LULUCF) & CO2 & National Total excluding LULUCF & absolute emissions\\
\hline
3 & EN.GHG.METH.KT.CE & Greenhouse gas emissions: Methane (CH4) (Total excluding LULUCF) & CH4 & National Total excluding LULUCF & absolute emissions\\
\hline
4 & EN.GHG.NOXE.KT.CE & Greenhouse gas emissions: Nitrous Oxide (N2O) (Total excluding LULUCF) & N2O & National Total excluding LULUCF & absolute emissions\\
\hline
5 & EN.GHG.FGAS.KT.CE & Greenhouse gas emissions: Fluorinated Gases (Total excluding LULUCF) & FGASES (AR4GWP100) & National Total excluding LULUCF & absolute emissions\\
\hline
6 & EN.GHG.TOTL.AG.KT.CE & Greenhouse gas emissions: All Kyoto Gases (Sector = Agriculture) & KYOTOGHG (AR4GWP100) & Agriculture & absolute emissions\\
\hline
7 & EN.GHG.TOTL.EG.KT.CE & Greenhouse gas emissions: All Kyoto Gases (Sector = Energy) & KYOTOGHG (AR4GWP100) & Energy & absolute emissions\\
\hline
8 & EN.GHG.TOTL.IN.KT.CE & Greenhouse gas emissions: All Kyoto Gases (Sector = Industrial Processes and Product Use) & KYOTOGHG (AR4GWP100) & Industrial Processes and Product Use & absolute emissions\\
\hline
9 & EN.GHG.TOTL.OT.KT.CE & Greenhouse gas emissions: All Kyoto Gases (Sector = Other) & KYOTOGHG (AR4GWP100) & Other & absolute emissions\\
\hline
10 & EN.GHG.TOTL.WA.KT.CE & Greenhouse gas emissions: All Kyoto Gases (Sector = Waste) & KYOTOGHG (AR4GWP100) & Waste & absolute emissions\\
\hline
11 & EN.GHG.CO2E.AG.KT & Greenhouse gas emissions: Carbon Dioxide (CO2) (Sector = Agriculture) & CO2 & Agriculture & absolute emissions\\
\hline
12 & EN.GHG.CO2E.EG.KT & Greenhouse gas emissions: Carbon Dioxide (CO2) (Sector = Energy) & CO2 & Energy & absolute emissions\\
\hline
13 & EN.GHG.CO2E.IN.KT & Greenhouse gas emissions: Carbon Dioxide (CO2) (Sector = Industrial Processes and Product Use) & CO2 & Industrial Processes and Product Use & absolute emissions\\
\hline
14 & EN.GHG.CO2E.OT.KT & Greenhouse gas emissions: Carbon Dioxide (CO2) (Sector = Other) & CO2 & Other & absolute emissions\\
\hline
15 & EN.GHG.CO2E.WA.KT & Greenhouse gas emissions: Carbon Dioxide (CO2) (Sector = Waste) & CO2 & Waste & absolute emissions\\
\hline
16 & EN.GHG.METH.AG.KT.CE & Greenhouse gas emissions: Methane (CH4) (Sector = Agriculture) & CH4 & Agriculture & absolute emissions\\
\hline
17 & EN.GHG.METH.EG.KT.CE & Greenhouse gas emissions: Methane (CH4) (Sector = Energy) & CH4 & Energy & absolute emissions\\
\hline
18 & EN.GHG.METH.IN.KT.CE & Greenhouse gas emissions: Methane (CH4) (Sector = Industrial Processes and Product Use) & CH4 & Industrial Processes and Product Use & absolute emissions\\
\hline
19 & EN.GHG.METH.OT.KT.CE & Greenhouse gas emissions: Methane (CH4) (Sector = Other) & CH4 & Other & absolute emissions\\
\hline
20 & EN.GHG.METH.WA.KT.CE & Greenhouse gas emissions: Methane (CH4) (Sector = Waste) & CH4 & Waste & absolute emissions\\
\hline
21 & EN.GHG.NOXE.AG.KT.CE & Greenhouse gas emissions: Nitrous Oxide (N2O) (Sector = Agriculture) & N2O & Agriculture & absolute emissions\\
\hline
22 & EN.GHG.NOXE.EG.KT.CE & Greenhouse gas emissions: Nitrous Oxide (N2O) (Sector = Energy) & N2O & Energy & absolute emissions\\
\hline
23 & EN.GHG.NOXE.IN.KT.CE & Greenhouse gas emissions: Nitrous Oxide (N2O) (Sector = Industrial Processes and Product Use) & N2O & Industrial Processes and Product Use & absolute emissions\\
\hline
24 & EN.GHG.NOXE.OT.KT.CE & Greenhouse gas emissions: Nitrous Oxide (N2O) (Sector = Other) & N2O & Other & absolute emissions\\
\hline
25 & EN.GHG.NOXE.WA.KT.CE & Greenhouse gas emissions: Nitrous Oxide (N2O) (Sector = Waste) & N2O & Waste & absolute emissions\\
\hline
26 & EN.GHG.FGAS.IN.KT.CE & Greenhouse gas emissions: Fluorinated Gases (Sector = Industrial Processes and Product Use) & FGASES (AR4GWP100) & Industrial Processes and Product Use & absolute emissions\\
\hline
27 & EN.GHG.CO2E.LU.KT.CE & Greenhouse gas emissions: Carbon Dioxide (CO2) (Sector = LULUCF) & CO2 & Land Use, Land Use Change, and Forestry & absolute emissions\\
\hline
28 & EN.GHG.METH.LU.KT.CE & Greenhouse gas emissions: Methane (CH4) (Sector = LULUCF) & CH4 & Land Use, Land Use Change, and Forestry & absolute emissions\\
\hline
29 & EN.GHG.NOXE.LU.KT.CE & Greenhouse gas emissions: Nitrous Oxide (N2O) (Sector = LULUCF) & N2O & Land Use, Land Use Change, and Forestry & absolute emissions\\
\hline
30 & EN.GHG.TOTL.PC & Greenhouse gas emissions: All Kyoto Gases (Total excluding LULUCF) per capita & KYOTOGHG (AR4GWP100) & National Total excluding LULUCF & per capita\\
\hline
31 & EN.GHG.CO2E.PC & Greenhouse gas emissions: Carbon Dioxide (CO2) (Total excluding LULUCF) per capita & CO2 & National Total excluding LULUCF & per capita\\
\hline
32 & EN.GHG.TOTL.PP.GD & Greenhouse gas emissions: All Kyoto Gases (Total excluding LULUCF) per 2017 PPP \$ of GDP & KYOTOGHG (AR4GWP100) & National Total excluding LULUCF & per GDP\\
\hline
33 & EN.GHG.CO2E.PP.GD & Greenhouse gas emissions: Carbon Dioxide (CO2) (Total excluding LULUCF) per 2017 PPP \$ of GDP & CO2 & National Total excluding LULUCF & per GDP\\
\hline
34 & EN.GHG.TOTL.KD.GD & Greenhouse gas emissions: All Kyoto Gases (Total excluding LULUCF) per 2017 US\$ of GDP & KYOTOGHG (AR4GWP100) & National Total excluding LULUCF & per GDP\\
\hline
35 & EN.GHG.CO2E.KD.GD & Greenhouse gas emissions: Carbon Dioxide (CO2) (Total excluding LULUCF) per 2017 US \$ of GDP & CO2 & National Total excluding LULUCF & per GDP\\
\hline
36 & EN.GHG.TOTL.ZG & Greenhouse gas emissions: All Kyoto Gases (Total excluding LULUCF) \% change from 1990 & KYOTOGHG (AR4GWP100) & National Total excluding LULUCF & \% change\\
\hline
37 & EN.GHG.CO2E.ZG & Greenhouse gas emissions: Carbon Dioxide (CO2) (Total excluding LULUCF) \% change from 1990 & CO2 & National Total excluding LULUCF & \% change\\
\hline
38 & EN.GHG.METH.ZG & Greenhouse gas emissions: Methane (CH4) (Total excluding LULUCF) \% change from 1990 & CH4 & National Total excluding LULUCF & \% change\\
\hline
39 & EN.GHG.N2OX.ZG & Greenhouse gas emissions: Nitrous Oxide (N2O) (Total excluding LULUCF) \% change from 1990 & N2O & National Total excluding LULUCF & \% change\\
\hline
40 & EN.GHG.CO2E.ZS & Greenhouse gas emissions: Carbon Dioxide (CO2) (Total excluding LULUCF) share of total GHG emissions & CO2 & National Total excluding LULUCF & share of total\\
\hline
41 & EN.GHG.METH.ZS & Greenhouse gas emissions: Methane (CH4) (Total excluding LULUCF) share of total GHG emissions & CH4 & National Total excluding LULUCF & share of total\\
\hline
42 & EN.GHG.N2OX.ZS & Greenhouse gas emissions: Nitrous Oxide (N2O) (Total excluding LULUCF) share of total GHG emissions & N2O & National Total excluding LULUCF & share of total\\
\hline
43 & EN.GHG.FGAS.ZS & Greenhouse gas emissions: Fluorinated Gases (Total excluding LULUCF) share of total GHG emissions & FGASES (AR4GWP100) & National Total excluding LULUCF & share of total\\
\hline
\end{tabu}

\hypertarget{description-of-data-source}{%
\subsection{Description of data
source}\label{description-of-data-source}}

The PRIMAP-hist dataset, developed and published by the Potsdam
Institute for Climate Impact Research (PIK, see
https://www.pik-potsdam.de/en/home) in Germany, a non-profit
organization conducting research on climate change, contains historical
greenhouse gas emissions since 1750. The dataset includes emissions
aggregated by country (200+ countries), category/sector (IPCC 2006
categories for emissions) and greenhouse gas/entity (all Kyoto GHGs).
The dataset is regularly updated and currently has a lag of one (1) year
(data available until 2022, published in October 2023). The dataset is
compiled from several other public data sets (see methodology).

The methodology to make the data from various sources comparable is
described in this paper: Gütschow, J.; Jeffery, L.; Gieseke, R.; Gebel,
R.; Stevens, D.; Krapp, M.; Rocha, M. (2016): The PRIMAP-hist national
historical emissions time series, Earth Syst. Sci. Data, 8, 571-603,
doi:10.5194/essd-8-571-2016.

The dataset is compiled using data from the following sources:

\begin{itemize}
\tightlist
\item
  Global CO2 emissions from cement production v220919 (Andrew 2022)
\item
  BP Statistical Review of World Energy website: Energy Institute (2023)
\item
  CDIAC data: Boden et al.~(2017): Gilfillan et al.~(2020), paper
  Gilfillan and Marland (2021)
\item
  EDGAR version 7.0: data, website, Reports: JRC (2022), JRC (2021)
\item
  EDGAR-HYDE 1.4 data: Van Aardenne et al.~(2001), Olivier and Berdowski
  (2001)
\item
  FAOSTAT database data: Food and Agriculture Organization of the United
  Nations (2023)
\item
  RCP historical data data, paper: Meinshausen et al.~(2011)
\item
  UNFCCC National Communications and National Inventory Reports for
  developing countries
\item
  UNFCCC Biennial Update Reports, National Communications, and National
  Inventory Reports for developing countries
\item
  UNFCCC Common Reporting Format (CRF)
\item
  Official country repositories (non-UNFCCC) for Taiwan, South Korea
\end{itemize}

\hypertarget{relevance-to-development}{%
\subsection{Relevance to development}\label{relevance-to-development}}

Global warming and climate change impact development in a multitude of
ways. Climate change changes the availability of water, the prevalence
of extreme weather events such as floods and droughts and warms and
rises the oceans. This has immediate impacts on food security, health,
poverty, displacement and biodiversity. Greenhouse gas emissions
contribute to global warming and therefore measures of greenhouse gas
emissions, by greenhouse gas, by country, over time and by sector is
pivotal to understanding climate change.

\hypertarget{high-quality}{%
\subsection{High quality}\label{high-quality}}

\hypertarget{source}{%
\subsubsection{Source}\label{source}}

The PRIMAP-hist data series is disseminated under the \textbf{Creative
Commons Attribution 4.0 International (CC BY 4.0) license}. This license
allows to redistribute the material in any medium or format and adapt as
well, as long as appropriate credit is given and indicate if any changes
were made.

The most recent release was on October 15, 2023. Since the 2022 release,
the time lag is only one year.

Latest version, released October 15, 2023: \textbf{Gütschow, J., \&
Pflüger, M. (2023). The PRIMAP-hist national historical emissions time
series (1750-2022) v2.5 (2.5) {[}Data set{]}. Zenodo.
https://doi.org/10.5281/zenodo.10006301}

The first version of the dataset was released in 2016, with regular
releases since.

Earlier versions:

Released Mar 15, 2023: Gütschow, J.; Pflüger, M. (2023): The PRIMAP-hist
national historical emissions time series v2.4.2 (1750-2021). zenodo.
doi:10.5281/zenodo.7727475.

Released Feb 20, 2023: Gütschow, J.; Pflüger, M. (2023): The PRIMAP-hist
national historical emissions time series v2.4.1 (1750-2021). zenodo.
doi:10.5281/zenodo.7585420.

Released October 17, 2022: Gütschow, J.; Pflüger, M. (2022): The
PRIMAP-hist national historical emissions time series v2.4 (1750-2021).
zenodo. doi:10.5281/zenodo.7179775.

Released September 22, 2021: Gütschow, J.; Günther, A.; Pflüger, M.
(2021): The PRIMAP-hist national historical emissions time series
(1750-2019). v2.3.1. zenodo. https://doi.org/10.5281/zenodo.5494497

Released August 30, 2021: Gütschow, J.; Günther, A.; Pflüger, M. (2021):
The PRIMAP-hist national historical emissions time series v2.3
(1850-2019). zenodo. doi:10.5281/zenodo.5175154.

Released February 9, 2021: Gütschow, J.; Günther, A.; Jeffery, L.;
Gieseke, R. (2021): The PRIMAP-hist national historical emissions time
series v2.2 (1850-2018). zenodo. doi:10.5281/zenodo.4479172.

Gütschow, Johannes; Jeffery, Louise; Gieseke, Robert; Günther, Annika
(2019): The PRIMAP-hist national historical emissions time series
(1850-2017). v2.1. GFZ Data Services.
https://doi.org/10.5880/PIK.2019.018.

Gütschow, J., Jeffery, M. L., Gieseke, R., Gebel, R., Stevens, D.,
Krapp, M., and Rocha, M. (2016): The PRIMAP-hist national historical
emissions time series, Earth Syst. Sci. Data, 8, 571--603,
https://doi.org/10.5194/essd-8-571-2016.

Using one source for all GHG indicators guarantees comparability across
indicators.

\hypertarget{unique-visitors}{%
\subsubsection{Unique visitors}\label{unique-visitors}}

No data on number of unique visitors for new series. The latest version
of the PRIMAP-hist data was dowmloaded 14,000 times in the first month
after publication. The data is also available on the
\href{https://www.climatewatchdata.org/ghg-emissions}{Climate Watch
(CAIT)} website.

Improved data quality and coverage can increase the number of unique
visitors of the GHG indicators in WDI.

\hypertarget{methodology}{%
\subsection{Methodology}\label{methodology}}

Copied from https://zenodo.org/record/7585420

The PRIMAP-hist dataset combines several published datasets to create a
comprehensive set of greenhouse gas emission pathways for every country
and Kyoto gas, covering the years 1750 to 2021, and almost all UNFCCC
(United Nations Framework Convention on Climate Change) member states as
well as most non-UNFCCC territories. The data resolves the main IPCC
(Intergovernmental Panel on Climate Change) 2006 categories. For CO2,
CH4, and N2O subsector data for Energy, Industrial Processes and Product
Use (IPPU), and Agriculture are available. The ``country reported data
priority'' (CR) scenario of the PRIMAP-hist datset prioritizes data that
individual countries report to the UNFCCC. For developed countries,
AnnexI in terms of the UNFCCC, this is the data submitted anually in the
``common reporting format'' (CRF). For developing countries, non-AnnexI
in terms of the UNFCCC, this is the data available through the UNFCCC DI
interface (di.unfccc.int) with additional country submissions read from
pdf and where available xls(x) or csv files. For a list of these
submissions please see below. For South Korea the 2021 official GHG
inventory has not yet been submitted to the UNFCCC but is included in
PRIMAP-hist. PRIMAP-hist also includes official data for Taiwan which is
not recognized as a party to the UNFCCC.

Gaps in the country reported data are filled using third party data such
as CDIAC, BP (fossil CO2), Andrew cement emissions data (cement),
FAOSTAT (agriculture), and EDGAR v7.0 (all sectors). Lower priority data
are harmonized to higher priority data in the gap-filling process.

For the third party priority time series gaps in the third party data
are filled from country reported data sources.

Data for earlier years which are not available in the above mentioned
sources are sourced from EDGAR-HYDE, CEDS, and RCP (N2O only) historical
emissions.

The v2.4 release of PRIMAP-hist reduced the time-lag from 2 to 1 years.
Thus we include data for 2021 while the 2.3.1 version included data for
2019 only. For energy CO\(_2\) growth rates from the BP statistical
review of world energy are used to extend the country reported (CR) or
CDIAC (TP) data to 2021. For CO\(_2\) from cement production Andrew
cement data are used. For other gases and sectors, EDGAR 7.0 is used in
PRIMAP-hist v2.4.1 (v2,4 had to rely on numerical methods ).

Version 2.4.1 of the PRIMAP-hist dataset does not include emissions from
Land Use, Land-Use Change, and Forestry (LULUCF) in the main file.
LULUCF data are included in the file with increased number of
significant digits and have to be used with care as they are constructed
from different sources using different methodologies and are not
harmonized.

Notes

\begin{itemize}
\tightlist
\item
  Emissions from international aviation and shipping are not included in
  the dataset.
\item
  Emissions from Land Use, Land-Use Change, and Forestry (LULUCF) are
  not included in the main version of this dataset. They are included in
  the version without rounding as users need to take extra care when
  using LULUCF data because changes some of the year-to-year changes in
  the data come from using different sources or methodology changes
  within a source rather than changes in actual emissions.
\end{itemize}

\hypertarget{adequate-coverage}{%
\subsection{Adequate coverage}\label{adequate-coverage}}

For the PRIMAP-hist data coverage is the same for all indicators, as
data has been imputed where necessary.

For the current indicators, coverage depends on the source of the
indicator.

\hypertarget{number-of-economies}{%
\subsubsection{Number of economies}\label{number-of-economies}}

\textbf{PRIMAP-hist}

The WDI includes 217 countries of which 198 are included in the
PRIMAP-hist dataset. The country coverage for the PRIMAP-hist data is 91
percent.

The following WDI countries are not included in the PRIMAP-hist dataset.
Note that most of these countries are overseas territories, which are
likely included in the total of the country or economy they belong to
(exceptions are West Bank and Gaza and Kosovo).

\begin{tabular}{l|l}
\hline
long\_name & ISO3\\
\hline
American Samoa & ASM\\
\hline
Bermuda & BMU\\
\hline
Cayman Islands & CYM\\
\hline
Channel Islands & CHI\\
\hline
Curacao & CUW\\
\hline
Faroe Islands & FRO\\
\hline
French Polynesia & PYF\\
\hline
Gibraltar & GIB\\
\hline
Greenland & GRL\\
\hline
Guam & GUM\\
\hline
Isle of Man & IMN\\
\hline
Kosovo & XKX\\
\hline
New Caledonia & NCL\\
\hline
Northern Mariana Islands & MNP\\
\hline
Puerto Rico & PRI\\
\hline
Sint Maarten (Dutch part) & SXM\\
\hline
St. Martin (French part) & MAF\\
\hline
Virgin Islands (U.S.) & VIR\\
\hline
West Bank and Gaza & PSE\\
\hline
\end{tabular}

\hypertarget{share-of-low-and-middle-income-countries}{%
\subsubsection{Share of low and middle income
countries}\label{share-of-low-and-middle-income-countries}}

\textbf{PRIMAP-hist}

The PIK data include 132 LMIC countries, which is 99 percent of the
total LMIC countries in WDI.

\hypertarget{absolute-latest-year}{%
\subsubsection{Absolute latest year}\label{absolute-latest-year}}

\textbf{PRIMAP-hist}

The absolute latest year in the PIK data is 2022.

\hypertarget{median-latest-year}{%
\subsubsection{Median latest year}\label{median-latest-year}}

\textbf{PRIMAP-hist}

The median latest year in the PIK data is 2022. This is the same for the
data without extrapolation.

\hypertarget{span-of-years}{%
\subsubsection{Span of years}\label{span-of-years}}

\textbf{PRIMAP-hist}

The span of years in the PIK data is 63. Data is available from 1750,
but this is not relevant for the WDI.

\hypertarget{non-missing-data}{%
\subsubsection{Non-missing data}\label{non-missing-data}}

Time and country coverage of existing indicators showing that latest
year of many indicators is 2014 or 2018 and country coverage is low for
some indicators.

\hypertarget{additional-issuesquestions-for-consideration}{%
\subsection{Additional issues/questions for
consideration}\label{additional-issuesquestions-for-consideration}}

\hypertarget{issues-with-current-data-sourcesindicators}{%
\subsubsection{Issues with current data
sources/indicators}\label{issues-with-current-data-sourcesindicators}}

The CAIT data are more volatile and the volatility does not always have
explaining factors. Volatility partially occurs because of the way the
series is compiled by CAIT. CAIT (and other sources) compile their
series using different primary sources. The sources can be broadly
divided into country-reported data and third-party data. Differences
between these two sources can be large and CAIT fills missing data from
one source with missing data from another sources, which explains in
part the observed volatility in the data. CAIT does not use official
inventories reported to UNFCCC, PIK does for countries where available.

Licensing issues and data availability: indicators from IEA are
distributed under a license that restricts redistribution and only
available until 2014.

The official SDG indicators on GHG emissions are

\begin{itemize}
\tightlist
\item
  INDICATOR 9.4.1 CO2 emission per unit of value added and;
\item
  INDICATOR 13.2.2 Total greenhouse gas emissions per year
\end{itemize}

See https://unstats.un.org/sdgs/dataportal/database. The respective
sources are UNIDO and UNFCCC.



\end{document}
